%% Code source du rapport de 3e année
%% Copyright (C) 2016  Ludovic Hondet

%% This program is free software: you can redistribute it and/or modify
%% it under the terms of the GNU General Public License as published by
%% the Free Software Foundation, either version 3 of the License, or
%% (at your option) any later version.

%% This program is distributed in the hope that it will be useful,
%% but WITHOUT ANY WARRANTY; without even the implied warranty of
%% MERCHANTABILITY or FITNESS FOR A PARTICULAR PURPOSE.  See the
%% GNU General Public License for more details.
				  
%% You should have received a copy of the GNU General Public License
%% along with this program.  If not, see <http://www.gnu.org/licenses/>.

\documentclass[12pt]{report}
\usepackage{fontspec,polyglossia}
\usepackage{calc,layouts}
\usepackage[inner=25mm,hmarginratio=2:3,textheight=678pt,twoside,a4paper,
  marginparwidth=80pt,headheight=15pt]{geometry}
\usepackage{array,tabulary,multirow,graphicx}
\usepackage[autostyle=once]{csquotes}
\usepackage{titlesec}
\usepackage{paralist}
\usepackage{amsmath,eurosym}
\usepackage{floatrow,rotating}
\usepackage{wrapfig}
\usepackage{fancyhdr}
\usepackage{makeidx}
\usepackage[backend=bibtex8,natbib=true,style=authoryear]{biblatex}
\usepackage[verbose,hyperindex=false]{hyperref}
\usepackage[xindy,acronym,nomain,nohypertypes={acronym}]{glossaries}


\hypersetup{colorlinks=true,
  linkcolor=blue,
  citecolor=blue,
  pdfpagemode=UseOutlines,
  pdfinfo={
    Title={Mémoire de fin d'études},
    Author={Ludovic Hondet},
    Keywords={bois-énergie, mobilisation, transition énergétique, politique publique, propriétaires forestiers}
  }
}

\titleformat{\chapter}[hang]{\bfseries\LARGE\sffamily}{\thechapter}{.5em}{}
\floatsetup[table]{footposition=bottom,capposition=top}
\floatsetup[sidewaystable]{capposition=top}
\setdefaultlanguage{french}
\setotherlanguage{english}
\graphicspath{{images/}}
\setmainfont[Ligatures=TeX]{Latin Modern Roman}
\providecommand{\keywords}[1]{\textbf{Mots clefs~:}#1}
\providecommand{\enKeywords}[1]{\textbf{Keywords:}#1}

\bibliography{biblio}
\defbibheading{bibliog}{
  \chapter*{Bibliographie}
  \markboth{#1}{#1}
  \addcontentsline{toc}{chapter}{\protect\numberline{}Bibliographie}
}

\newcounter{table1}
\newcommand\indexp[1]{#1\index{#1}}
\makeindex
\makeglossaries
\input{glossaireEtAbbrv}


\fancyhf{}
\fancyhead[LO,RE]{\thepage}
\fancyhead[RO,LE]{\nouppercase{\leftmark}}

\fancypagestyle{plain}{
  \fancyhf{}
  \fancyhead[LO,RE]{\thepage}
  \renewcommand\headrulewidth{0pt}
}

\title{Quand la transition énergétique redéfinit\\
la gestion forestière\\
{\normalsize Enquête exploratoire
sur un programme de mobilisation des bois en Gironde}}
\author{Ludovic Hondet}
\date{\today}

\begin{document}


\newgeometry{top=20mm,bottom=20mm,left=25mm,right=25mm}
\begin{titlepage}    
    \centering
    {\normalsize Ministère de l'Agriculture, de l'Agroalimentaire et de la Forêt \par}
    \vspace{20.736pt}
    {{\LARGE École Nationale Supérieure des Sciences
    Agronomiques de Bordeaux Aquitaine}\\
    {\normalsize 1 cours du Général de Gaulle - CS 40201 - 33175 Gradignan CEDEX \par}}
    \vspace{20.736pt}
    {{\LARGE Mémoire de fin d'études}\\
    {\normalsize pour l'obtention du titre}\\
    {\LARGE Ingénieur de Bordeaux Sciences Agro} \par}
    \vspace{72.576pt}
    {{\LARGE \textbf{Quand la transition énergétique redéfinit\\
    la gestion forestière}} \par}
    \vspace{7.2pt}
    {{\Large Enquête exploratoire\\
    sur un programme de mobilisation des bois en Gironde} \par}
    \vspace{20.736pt}
    {{\normalsize When the energy transition redefines forestry management}\\
    {\normalsize Exploratory enquiry about a wood mobilisation programme in Gironde} \par}

    \vfill

    {\large \textit{Hondet, Ludovic} \par}
    \begin{flushleft}
      {\large Spécialisation: Foresterie\\
      Étude réalisé à: Institut national de Recherche en Sciences et Technologies pour l'Environnement et l'Agriculture\\
      50 avenue de Verdun Gazinet -- 33612 Cestas CEDEX \par}
    \end{flushleft}
    
    \vspace{1mm}
    
    {\includegraphics[width=20mm]{logo_irstea.jpg} \hfill \LARGE{- 2017 -} \hfill \includegraphics[width=20mm]{logo_BxScAgro.png}}
\end{titlepage}

\pagenumbering{gobble}
\newpage
\thispagestyle{empty}
~
\newpage

\restoregeometry
\maketitle

\newpage
\thispagestyle{empty}
~
\newpage

%\newpage
%\thispagestyle{empty}
%\setcounter{page}{0}
%\noindent Le fichier source de ce document est disponibles sur GitHub à l'adresse:\\
%\verb?https://github.com/LudovicHDT/FIRe_LaTeX?\\
%\noindent Les codes sources des scripts écrits pour ce rapport sont disponible à l'adresse:\\
%\verb?https://github.com/LudovicHDT/A7R?\\
%Si vous désirez des documents ou figures dont il est fait mention dans le rapport mais qui
%ne sont pas en annexes, vous pouvez vous adresser à l'auteur:\\
%\verb?ludohon@gmail.com?

%{\footnotesize
%\begin{verbatim}

%Copyright (C) 2017  Ludovic Hondet

%This document is free document: you can redistribute it and/or modify
%it under the terms of the GNU General Public License as published by
%the Free Software Foundation, either version 3 of the License, or
%(at your option) any later version.

%This document is distributed in the hope that it will be useful,
%but WITHOUT ANY WARRANTY; without even the implied warranty of
%MERCHANTABILITY or FITNESS FOR A PARTICULAR PURPOSE.  See the
%GNU General Public License for more details.

%You should have received a copy of the GNU General Public License
%along with this program.  If not, see <http://www.gnu.org/licenses/>.
%\end{verbatim}
%}
%\newpage

\pagestyle{fancy}
\renewcommand{\chaptermark}[1]{\markboth{#1}{}}

\tableofcontents
\listoffigures
\addcontentsline{toc}{chapter}{\protect\numberline{}Table des figures}
\listoftables
\addcontentsline{toc}{chapter}{\protect\numberline{}Liste des tableaux}
\printacronyms[title={Liste des abréviations},nonumberlist]
\addcontentsline{toc}{chapter}{\protect\numberline{}Liste des abréviations}


\chapter*{Préface - remerciements}
\addcontentsline{toc}{chapter}{\protect\numberline{}Préface - remerciements}

Trois ans ont passé depuis mon entrée en école d'ingénieur. Mon projet
professionnel à donc eu le temps mûrir. Je suis rentré à Bordeaux Sciences Agro
dans l'esprit de travailler plus tard dans le monde de la forêt. Les cours que
nous avons suivi ne ferons pas de nous des sylviculteurs experts du chêne~; à
la place, ils nous apprennent à analyser et comprendre les tenants et
aboutissants d'une filière, qu'une filière est lié à son environnent. Elle
dépend donc de l'économie, de la finance et des technologies qui l'entourent. À
ce titre, j'ai perçu la forêt comme un objet d'étude, comme une ressource à
laquelle on peut appliquer des méthodes pour mieux la gérer, la ressource et
toute la filière qui en découle. Plutôt que de me spécialiser dans la technique
au travers du stage, j'ai donc opté pour une poursuite d'études qui m'apportera
les connaissances nécessaires à la meilleure gestion d'une ressource dans son
environnement socio-économique.

C'est donc en toute logique que j'ai accepté un stage qui allie forêt et
politique publique~; stage qui assure une continuité entre formation d'ingénieur
agronome en spécialité foresterie et un mastère spécialisé Politique Publiques
et Stratégies pour l'Environnement à AgroParisTech\\

Je souhaite rappeler que, en tant que porteur du projet FOREDAVENIR, le CRPF a
accepté nous laisser accéder aux informations qui nous ont permis de contacter les
propriétaires. C'est donc grâce à ses agents que mon stage a pu se dérouler dans de
bonnes conditions. Je remercie donc Henri Husson, directeur adjoint du CRPF
d'Aquitaine~; Jérémy Abgrall, coordinateur du projet FOREDAVENIR et Dimitri
Blanchard, technicien CRPF. Je remercie également mon maître de stage, Vincent
Banos, pour ces relectures attentives et Philippe Deuffic pour ses précieux
conseils et sa réflexion sur la typologie.

Enfin, je remercie toutes les personnes interrogées~: les opérateurs et les
services de l'État qui ont toujours accepté de répondre à toutes les questions,
même les plus provocatrices, ainsi
que les propriétaires qui ont, bien souvent, accepté de sacrifier un peu de
temps libre pour participer aux entretiens. 

\newpage
\thispagestyle{empty}
~
\newpage

\chapter{Introduction}
\pagenumbering{arabic}

Le 12 décembre 2015, l'Accord de Paris sur le climat est signé par 194 pays et
l'Union Européenne. L'ambition affichée est de contenir le réchauffement
climatique en dessous des 2$^{\circ}$C. Pour atteindre cet objectif, les partenaires signataires
se sont engagées à accentuer la transformation des modes de production et de
consommation d'énergie. Ils donc incitent les acteurs de l'énergie, les particuliers
et les entreprises à passer d'un modèle basé sur les énergies fossiles à un
modèle décarboné. C'est ce qui s'appelle la transition énergétique
\citep{kern2008_ref112,rojey2008_ref118}. En France, la transition
énergétique se matérialise par l'adoption d'une loi \textquote{transition
énergétique pour la croissance verte}. Celle-ci prévoit de porter la part des
énergies renouvelables à 23\% de la consommation d'énergie brute d'ici 2020. Les
usages énergétiques du bois s'inscrivent dans une longue histoire, les
technologies du~\gls{be} peuvent apparaître mieux maîtrisés que les
technologies solaires et éoliennes~\citep{bontoux2009_ref119}. Or, la ressource forestière est
décrite comme abondante~\citep{ifn2005_ref120}. De plus, sa valorisation énergétique est
propice au développement local grâce à la proximité des approvisionnements
\citep{poinsot2012_ref122,tritz2012_ref123}. Le BE s'est donc imposé comme un
moteur incontournable de cette trajectoire de
diversification du mix énergétique~\citep{ademeMixEn_ref108}. En effet, le BE
constitue la première source d'énergie renouvelable, loin devant l'hydraulique
(44\% contre 26\%). En conséquence, il est prévu que la biomasse contribue pour
moitié aux objectifs de 23\% d'énergie renouvelable d'ici 2020.\\

L'usage du bois comme moyen pour atteindre les objectifs de la transition
énergétique ne doit cependant pas éluder des caractéristique de la forêt et de
ses filières. D'une part, les ressources forestières sont parfois difficilement
accessibles (zones de montagne, propriété morcelée, etc.)~; d'autre part, la
matière première mobilisée pour le BE est déjà utilisée par la filière~\gls{bi}.
Dans certaines zones comme dans les Landes de Gascogne, cette concurrence d'usages
a pu devenir critique suite aux tempêtes. L'utilisation de la ressource
locale en tant que BE a donc suscité une levée de boucliers~\citep{alexandre2012_ref124,ecofor2010_ref125}. Ces écueils
traduisent des différences de préoccupations et de culture entre les mondes de
l'énergie et de la forêt. Ils expliquent, en partie, pourquoi plus de 30 ans
après l'installation des premières chaufferies BE, on observe des difficultés
persistantes au niveau de la filière BE~\citep{tabourdeau2014_ref117,BEbanosDehez_ref78}.
Toutefois, des signaux expriment une appropriation des enjeux de transition
énergétique et une transformation de la menace du \gls{be} en opportunité de
développement par la filière forêt-bois \citep{sergent2014_ref110}.
Dans le cadre du~\gls{pnfb}\index{PNFB}~\citep{pnfb_ref98}, l'État s'est engagé
à dynamiser la gestion forestière et à augmenter les prélèvements de 12 millions de
m$^{3}$ de bois supplémentaire d'ici 2026 afin, entre autre, de contribuer à la
transition bas carbone. Par ailleurs,~\gls{ademe}\index{ADEME}, acteur historique
du développement des énergies renouvelables dont le BE, a lancé en 2015 les
\glspl{ami}\index{AMI}. Ceux-ci ont pour
but de faire émerger des projets collaboratifs favorisant la mobilisation de
bois additionnels pour les chaufferies biomasse du Fonds Chaleur. FOREDAVENIR
est le projet retenu en Gironde suite à ces \glspl{ami}.\\

\indexp{FOREDAVENIR} est un programme porté par le~\gls{crpf}\index{CRPF}
d'Aquitaine. Ce programme, financé par le Fond Chaleur de l'ADEME, a pour
objectif de remettre en production 1395ha de \textquote[\citealp{instructionTech_ref109}]
{taillis, de~\gls{tsf} ou d'accrus}. Le CRPF a prévu d'atteindre cet objectif en incitant
les propriétaires à nettoyer leurs bois grâce à des subventions sur les travaux
de reboisement, de balivage, de cloisonnement ou de régénération naturelle. Le
CRPF envisage ainsi de mobiliser 167 670t de bois, dont 110 000t à destination
des chaufferies de Bordeaux métropole. Les forêts de l'Est-Gironde
ont la particularité d'être constituées en majorité de petites propriétés,
la surface moyenne par propriété est de 1.197ha~\citep{diagForedavenir_ref71}.
De plus, le CRPF d'Aquitaine est confronté à des
propriétaires qui ont une faible activité sylvicole et qui connaissent
très mal leurs bois. Le CRPF d'Aquitaine est donc face à des propriétaires,
dans l'ensemble, inactifs.\\

Pour mobiliser des propriétaires inactifs, les porteurs de politiques publiques
ont plusieurs outils à leur disposition. Pour adapter le programme et ainsi
faciliter son implémentation, les porteurs peuvent entreprendre une enquête de
terrain~\citep{IRSTEA_ref61}. L'enquête permet de récolter de l'information pour
mieux appréhender les logiques d'actions et les motivations des propriétaires.
L'organisation de l'information recueilli sous forme de typologie est efficace
pour s'assurer que le programme proposé concorde avec les attentes de la
population qui en bénéficiera~\citep{boon2004_ref101,majumdar2008_ref103}.
Les typologies de propriétaires sont construites selon l'information que l'on
veut obtenir, il est ainsi possible de faire des typologies de comportement, de
motivations ou d'action raisonnée~\citep{dayer2014_ref95}. Elles n'ont pas les
rôles et ne remplissent pas les mêmes objectifs. Les typologies de comportement
ou d'action raisonnée classent les propriétaires sur la base d'éléments factuels
tel que l'existence de plans de gestion~\citep{young2015_ref104,ross-davis2007_ref94}
ou les pratiquent sylvicoles~\citep{bohlin2002_ref96,selter2009_ref93,novais2010_ref92}.
Ces typologies permettent de contextualiser les actions des propriétaires. Cette
contextualisation, en faisant décrire au propriétaire ses pratiques et son but,
permet de faire émerger les raisons qui justifient ses actions. La construction
de telles typologies s'appliquent bien aux propriétaires forestiers. Certains
tiennent compte des enjeux environnementaux et de biodiversité. Ils adaptent
leurs pratiques en fonction de ces enjeux~\citep{häyrinen2014_ref102}. Les typologies de motivation
classent les propriétaires en tenant compte de leurs sources d'information~\citep{ross-davis2007_ref94},
de leurs loisirs~\citep{ross-davis2007_ref94,tian2015_ref105} ou des aides
financières\footnote{Aides de l’État ou défiscalisation}~\citep{tian2015_ref105,
halder2016_ref106}. A titre d'exemple, ces typologies permettent, lorsque l'on souhaite monter un
programme de mobilisation des bois ou de dynamisation de la sylviculture, de
savoir quels sont les moyens d'informations privilégiés des propriétaires
forestiers~\citep{foretEntreprise233_ref107}. Dans une seconde étape, la
connaissance des moyens d'informations permettra d'implémenter le programme plus
aisément.\\

On sait donc comment préparer l'implémentation d'une politique publique incitative.
La construction d'une typologie de comportement permet de comprendre les
logiques d'action des individus du groupe étudié. La construction d'une
typologie de motivations, permet d'estimer, à priori, quelles actions ou aides
vont inciter les individus à adhérer au programme. Cependant, aucune typologie
ne rend compte de l'évolution du comportement des individus face au programme.
En effet, elles utilisent des critères binaires comme adhérer à un plan de
gestion ou non, chasser ou non, etc.. De plus, mon stage se déroule après
le lancement du programme mais sans qu'il soit pour autant terminé. L'un des
intérêts du stage est donc de rendre compte
des impacts potentiels de FOREDAVENIR sur les propriétaires. Pour cela, je vais
donc construire une typologie dynamique qui utilise des critères graduels et
continus plutôt que des critères binaires.\\

Certes le programme, et donc par extension mon stage, place les propriétaires au
centre du dispositif mais ces derniers s'insèrent dans un environnement et
interagissent avec des acteurs de la filière forêt-bois. De plus, ce programme
a la particularité de s'appuyer sur plus de 10 partenaires, dont des partenaires
extérieurs au monde de la forêt. Donc, en tant que programme pas purement
forestier puisqu'il implique des forestiers et des énergéticiens, on peut se
demander si FOREDAVENIR va entraîner une évolution de l'ingénierie forestière.
On voit donc apparaître une double problématique qui peut se formuler comme
suit: Au delà des objectifs quantitatifs affichés, comment FOREDAVENIR va-t-il
modifier les pratiques et les cadres de la gestion forestière en Est-Gironde?\\

Pour répondre à cette problématique, nous avons choisi, avec mon maître de stage,
d'appréhender la gestion forestière à 2 échelles. Nous avons lors d'une première
phase, mené des enquêtes auprès des propriétaires afin de comprendre comment ces
derniers s'approprient FOREDAVENIR à l'échelle micro et, à l'échelle méso, auprès
des opérateurs afin de recontextualiser l'action des propriétaires tout en
identifiant les traces d'une possible évolution de l'ingénierie forestière.
Dans un premier temps, je présenterais donc le dispositif méthodologique. Je
présenterais ensuite les résultats de l'enquête, de l'échelle micro à l'échelle
méso avant de conclure en discutant des résultats et en montrant les perspectives.

%j'ai, dans une première phase et avec mon
%maître de stage, mené une
%série d'enquêtes parmi les propriétaires forestiers et les acteurs de la filière.
%Grâce aux données recueillis auprès des propriétaires, j'ai pu construire une
%typologie dynamique qui utilise des discriminants graduels et continus. En parallèles, j'ai
%retranscrit, en plus des discours des propriétaires, les récits des acteurs de
%la filière. J'ai ainsi pu faire émerger, grâce à une analyse transversale, les
%enjeux qui font débat chez l'ensemble des opérateurs de FOREDAVENIR.
%Pour conclure, j'ai mis les résultasts en perspective avec d'autres travaux de
%recherche tout en montrant les limites et perspectives aux travaux présentés.


\chapter{Matériel et méthodes}\label{sec:M&M}

\section{Préparation des enquêtes}

\subsection{Choix et définition du territoire d'enquête}

L'offre de stage proposait une étude sur l'ensemble des territoires de
\indexp{FOREDAVENIR}: Haute Gironde, Libournais, \gls{siphem}\index{SIPHEM} et Bazadais (voir annexe~\ref{App:appxA}).
Parmi ces quatre territoires, le Bazadais a la particularité d'être intégré,
géographiquement et en terme d'organisation, dans le massif Landais. Ce dernier
a fait l’objet de plusieurs travaux de recherche récents
\citep{BEbanosDehez_ref78,brahic2017_ref114,dehez2017_ref115}, nous avons donc
choisi de ne pas le retenir et de centrer l'étude sur le SIPHEM, le
Libournais et la Haute-Gironde. Ces 3 territoires ont en commun une forte
présence de feuillus, l'industrie du bois en est quasiment absente. Au
delà de ces caractéristiques qui sont des marqueurs forts
du projet FOREDAVENIR, le travail de
prospection nous a également permis de constater des éléments
distinctifs. Le SIPHEM est, par exemple, un \gls{tepos}\index{TEPOS} sur lequel des petites chaufferies sont présentes
\citep{BEbanosDehez_ref78} alors que ces équipements sont absents des
territoires du Libournais et de la Haute-Gironde. Il semblait donc intéressant
d'étudier les effets des dynamiques locales sur l'implémentation du
programme FOREDAVENIR au travers de ce contraste territorial. La présence d'une
opposition entre homogénéité sylvicole et hétérogénéité des dynamiques locales
offrait un cadre intéressant pour mener l'enquête.

\subsection{Choix de la méthode d'enquête}

Dans le cadre du projet de recherche \indexp{TREFFOR} (TRansition Énergétique et mutations
de la Filière FOrêt-bois en Région Aquitaine), l'IRSTEA prévoit de mettre en œuvre
une enquête quantitative. Cette enquête a pour but de mesurer l’appétence individuelle des
propriétaires forestiers pour le~\gls{be} et d’analyser les pratiques de
gestion développées dans le cadre de la transition énergétique. Ce questionnaire
sera réalisé en partenariat avec le CRPF. Il pourra aider les porteurs de
FOREDAVENIR à évaluer les effets de leur programme expérimental et donner des
indications permettant de faciliter l’implémentation de futurs
programmes. Pour connaître les éléments pertinents à intégrer dans
cette enquête quantitative, il est nécessaire d’identifier les pratiques de
gestion forestière développées par la population cible, ses comportements
vis-vis du \gls{be} et  les facteurs individuels et collectifs qui
favorisent ou, au contraire, freinent l’appropriation d’un programme
expérimental tel que FOREDAVENIR. L'\indexp{enquête qualitative}
permet cela~\citep{IRSTEA_ref69} puisqu’elle vise justement à explorer et à
comprendre les logiques d’action, c'est-à-dire la manière dont les acteurs
agissent et interagissent. Cette première phase permet d'entériner les
hypothèses préliminaires et d'en faire émerger de nouvelles qui pourront être
intégrées dans l'enquête quantitative.

Une enquête qualitative est constituée d'entretiens qui peuvent être conduit
selon 3 modalités: entretien libre, semi-directif ou directif. Dans chacun de
mes entretiens, j'ai abordé plusieurs thèmes sur lesquelles nous avions
des connaissances. Pour chacun de ces thèmes, nous voulions laisser de la
liberté à la personne interrogée afin de faire émerger, si ils existaient, de
nouveaux enjeux. D'après~\citet{IRSTEA_ref69}, l'\indexp{entretien semi-directif}
était la modalité d'enquête la plus adaptée aux objectifs du stage.

\subsection{Choix et définition des enquêtés}

\subsubsection{Définition des enquêtés}

Initialement nous avions prévu d'enquêter parmi l'ensemble des propriétaires
forestiers du terrain d'enquête. Cet échantillonnage nous aurait permis
d'identifier les points de blocage qui empêchaient certains propriétaires de
participer à ce programme expérimental. Cependant, nous risquions d'être
confronté à des entretiens pauvres en informations. Il aurait été difficile de
faire parler des enquêtés sur des thématiques (\gls{be}, mobilisation des
bois, etc.) dont ils n'avaient pas forcément connaissance. Afin d’optimiser le temps
relativement court de l’enquête, nous avons donc préféré nous focaliser sur des
propriétaires qui avaient au moins montré un certain intérêt pour FOREDAVENIR.
Ce choix a permis de nous concentrer sur la diversité potentielle de
ces propriétaires, tant dans leurs motivations initiales que dans leurs manières
de s’approprier le programme porté par le CRPF. Par ailleurs, s’il est important
de comprendre les ressorts individuels qui guident l’action, le propriétaire
n’est pas un acteur isolé. Il s’insère dans un environnement et construit son
comportement en interaction avec d’autres acteurs. Parallèlement aux entretiens
menés auprès des propriétaires, nous avons donc mené des entretiens
complémentaires auprès de différents acteurs impliqués dans FOREDAVENIR
(opérateurs économiques, techniciens forestiers, élus\ldots{}) pour essayer
d’appréhender et de caractériser, en termes de convergences ou de points de
tensions, les dynamiques collectives dans lesquelles les propriétaires s’insèrent.
Plus exploratoire, cette phase d’entretiens complémentaires visait à identifier
les facteurs collectifs qui pouvaient favoriser ou freiner l’implication des
propriétaires forestiers dans le programme expérimental FOREDAVENIR.

\subsubsection{Construction du cadre échantillonnage}

Nous avons voulu construire un échantillon qui rend compte d'une diversité des
profils. Nous ne voulions pas créer un échantillon statistique représentatif de
la population étudiée. La littérature nous a appris que, souvent, les
propriétaires les plus actifs sont ceux  qui ont de grandes propriétés ou des
plans de gestion. Les territoires sur lesquels nous avons travaillé ont
subi des remembrements ou des programmes de dynamisation de la sylviculture. La
prise en compte du critère de participation à une action antérieure nous aurait
permis de tester l'influence des dynamiques territoriales. Nous avions donc
besoin des critères de surface, de participation à une action antérieure et la
répartition des propriétés sur les 3 territoires pour construire notre
échantillon.

%D'après la littérature, la taille de la propriété forestière est un indicateur
%du degré d'implication des propriétaires dans les questions de sylviculture et
%de filière forêt-bois. L'expérience empirique~\citep{novais2010_ref92} montre
%que l'exécution ou la délégation de l'ensemble ou d'une partie des travaux
%sylvicoles est associée aux grandes propriétés forestières lorsque ces travaux
%s'insèrent dans un \indexp{itinéraire sylvicole}\footnote{Un itinéraire
%sylvicole est l'ensemble des travaux à réaliser pour arriver à l'objectif pour
%un peuplement spécifique}.

\subsubsection{Sélection opérationnelle des propriétaires}

Dans le cadre de mon stage, le programme FOREDAVENIR est porté par le \indexp{CRPF}, je
me suis donc appuyé sur lui pour obtenir une liste de personnes à contacter.
Pour des raisons d'anonymat et de sécurité de l'information, le CRPF ne pouvait
nous fournir que la surface des propriétés avec leur territoire d'appartenance
et un numéro d'identification. Afin d'appliquer au mieux notre schéma
d'échantillonnage, les propriétaires de la liste du CRPF ont été classés
dans 3 classes de surface: de 0 à 4, de 4 à 25,
et 25 et +. La technique d’échantillonnage qui a été utilisée s'apparente à un
\indexp{\'{e}chantillonnage par quotas}, c'est-à-dire que la proportion de
propriétaires à interroger par classe dans notre échantillon correspond à la proportion de
propriétaires par classes dans notre population (voir tableau~\ref{tab:ech}).\\

\begin{table}[t]
  \centering
  \caption{Répartition des propriétaires participant à FOREDAVENIR par classes
  de surface}\label{tab:ech}
  \begin{tabulary}{\columnwidth}{|l|c|c|c|c|c|}
    \hline Classes (ha) & ]0;4]&]4;25]&]25;+$\infty$]&Total\\
    \hline Nombre & 95 & 127 & 20 & 242\\
    \hline Pourcentage & 39.26\% & 52.48\% & 8.26\% & 100\%\\
    \hline
  \end{tabulary}
\end{table}

Sur la base de cette première typologie succincte, nous avons
indiqué au CRPF que nous voudrions 33 coordonnées de propriétaires pour les
contacter. Pour les sélectionner nous avons respecté les pourcentages par classes
de surface et nous avons fait attention à avoir approximativement le même nombre
de propriétaires par territoire\footnote{Au lancement du projet TEPOS, Pellegrue
et Auriolles étaient dans le SIPHEM~; pour l'étude des dynamiques, on considère
donc que Pellegrue et Auriolles font toujours partie du SIPHEM}.

\subsection{Création des grilles d'entretiens}\index{grille d'entretiens}

La réalisation d'entretiens semi-directifs\index{entretien semi-directif}
suppose la construction de grilles
d'entretiens pour structurer et conduire les échanges
avec les personnes interrogées. J'ai créé mes grilles en me basant sur des
grilles de~\citet{consoreDeufficDehez_ref79,deuffic2012_ref91}. Une enquête a
pour objectif de répondre aux questions "Qui fait quoi?", "Comment?" et "Pourquoi?". On
retrouvera donc toujours, parmi chaque thème abordé, une partie qui fait décrire,
une partie qui a pour but d'appréhender le cheminement du répondant et une
partie qui a pour but de faire apparaître les motivations qui justifient les
actions des enquêtés.\\

J'ai interrogé 2 grandes populations : la population cible, les propriétaires
forestiers et une population complémentaires composée d'opérateurs économiques,
d'agents des collectivités et des services de l’État. Nous ne cherchions
pas les mêmes informations parmi ces 2 populations. J'ai construit 2 modèles de
grilles : une pour les propriétaires et une pour les opérateurs économiques (voir
annexes~\ref{App:appxB} et \ref{App:appxC}). Chacune des grilles aborde plusieurs thèmes. Les services
de l’État et les collectivités se différencient entre eux. Il était donc impossible de faire une
grille commune. Les grilles ont
été construites selon les informations que l'ont cherche à obtenir. A titre
d’exemple, la compréhension des logiques d'action des propriétaires nécessitait
d'obtenir des informations factuelles sur les pratiques sylvicoles passées et
actuelles, comment le propriétaire perçoit FOREDAVENIR et comment il envisage
l'avenir de sa forêt. Pour les services de l’État et les collectivités, je
préparais une liste de questions, j'introduisais la discussion en leur demandant
d'expliciter leur rôles et leurs missions.

\section{Réalisation et analyse des entretiens}

J'ai privilégié 2 modes de contact pour la prise de rendez-vous, soit par
téléphone soit par e-mail. J'ai obtenus des entretiens auprès de 30 personnes:\\
Propriétaires forestiers : 17\\
Entretiens complémentaires : 13
\begin{compactitem}
  \item 5 acteurs économiques de la filière
  \item 8 agents des collectivités ou des services de l’État
\end{compactitem}
J'ai effectué les entretiens soit seul, soit avec Vincent, soit avec Philippe.
Sur une liste de 33 propriétaires, 16 ont accepté un entretien, 8 ont décliné
et 9 n'ont pas pu être contacté, soit 48, 24 et 27,27\%. J'ai
fait l'entretien d'un propriétaire supplémentaire car ce dernier avait un statut
particulier. La répartition des propriétaires par classes de surface a été
résumé dans le tableau~\ref{tab:ech2}.\\

\begin{table}[t]
  \centering
  \caption{Répartition des propriétaires rencontrés par classes de surface}\label{tab:ech2}
  \begin{tabulary}{\columnwidth}{|l|c|c|c|c|c|}
    \hline Classes (ha) & ]0;4]&]4;25]&]25;+$\infty$]&Total\\
    \hline Nombre & 5 & 8 & 3 & 16\\
    \hline Pourcentage & 31.3\% & 50.1\% & 18.8\% & 100\%\\
    \hline
  \end{tabulary}
  \floatfoot{Les proportions par classes de surface des propriétaires rencontrés
  correspondent aux proportions par classes de surface des propriétaires ayant
  répondu à FOREDAVENIR (voir tableau~\ref{tab:ech}).}
\end{table}

L'ensemble de ces entretiens ont eu lieu alors qu'aucune opération sylvicole
n'avait commencé. L'exécution des entretiens à ce stade de
FOREDAVENIR présentait des inconvénients. On ne pouvait saisir ni
l’ensemble de la démarche proposée par ce programme expérimental, ni ses
résultats. \textit{A contrario}, cette temporalité offrait la possibilité de
percevoir plus finement les leviers d’action, les freins à l’engagement des
propriétaires et les dynamiques en cours et leurs impacts potentiels sur
les propriétaires. C'est ce qui m'a permis de construire une
typologie dynamique. En réalisant l’enquête à la fin du programme, on aurait en
effet pu connaître les travaux réalisés mais pas forcément les changements
intervenus car, par exemple, il aurait été difficile d’appréhender d’où partait
le propriétaire. Alors que là, on pouvait espérer mettre plus facilement en
évidence motivations et le cheminement suivi par les propriétaires, de ses
pratiques initiales aux types de travaux envisagés. Pour construire ma typologie
en tenant compte de cette dynamique, j'ai utilisé le taux d'activité et le degré
d'insertion des propriétaires comme discriminants. L'utilisation de variables
continues plutôt que discrètes m'a permis de rendre compte d'une évolution
probable du comportement forestier des propriétaires. Je me suis fortement
inspiré de~\citet{novais2010_ref92}~pour créer mes modèles de travaux (voir
tableau~\ref{tab:novais}).\\

%C’est pour essayer de saisir
%le changement en train de se faire que nous relevé le défi  de la construction
%d’une typologie  dynamique visant justement à représenter la manière dont les
%propriétaires enquêtés pouvaient potentiellement changer de catégorie avec la
%mise en œuvre de FOREDAVENOR.\\


%a apporté des avantages et des inconvénients. Grâce à cette
%temporalité, il était possible de percevoir les dynamiques en cours et l'impact
%de celles-ci sur les propriétaires, c'est ce qui m'a permis de construire une
%typologie dynamique. Pour cela, j'ai utilisé le taux d'activité et le degré
%d'insertion des propriétaires comme discriminants. L'utilisation de variables
%continues plutôt que discrètes m'a permis de rendre compte d'une évolution
%probable du comportement forestier des propriétaires. Je me suis fortement
%inspiré de~\citet{novais2010_ref92}~pour créer mes modèles de travaux (voir
%tableau~\ref{tab:novais}).

\begin{table}
  \centering
  \caption{Modèles de travaux~\citep{novais2010_ref92}}\label{tab:novais}
  \begin{tabulary}{\columnwidth}{lllll}
    \hline &&\multicolumn{3}{l}{Récolte}\\
    \cline{3-5} &&N/A&Internalisation&Externalisation\\
    \hline Entretien sous-bois&N/A&NN&NI&NE\\
	   &Internalisation&IN&II&IE\\
	   &Externalisation&EN&EI&EE\\
    \hline
  \end{tabulary}
\end{table}

%L'inconvénient majeur était que, parmi les propriétaires
%rencontrés, aucun n'avait déposé un dossier qui avait été accepté; je ne pouvais
%donc pas connaître, via les propriétaires, l'intégralité de la procédure de
%FOREDAVENIR, de la première lettre à la réalistaion des travaux.

A partir du 15$^{e}$ entretien, j'ai vu apparaître une redondance de
l'information. J'ai donc considéré avoir une bonne diversité de points de vue~;
ce qui est l'objectif premier lorsque l'on réalise une enquête par entretiens.
De plus, une vingtaine d' entretiens suffisent \textquote[\citealp{IRSTEA_ref69}]
{largement} pour créer une typologie. A cela s'ajoute le nombre initial de propriétaires
joignables fournis par le CRPF (33) et les contraintes de temps à respecter pour
réaliser l'analyse et rédiger le rapport. Ces facteurs ont fait que j'ai décidé
d’arrêter les entretiens de propriétaires à 17. Tous les entretiens ont été
retranscrits, soit sous la forme de synthèse à partir de la
prise de notes, soit ont été retranscrits partiellement ou totalement grâce aux
enregistrements audios. 6 entretiens ont été retranscrits totalement par la
transcriptrice IRSTEA, j'ai retranscrit partiellement les autres en relevant
les verbatims qui rendent compte des argumentaires développés. L'analyse s’est
ainsi déroulée en deux temps. Pour
chaque entretien, j’ai tout d’abord identifié les thèmes et enjeux structurants
pour l’enquêté et les argumentaires associés.  Ensuite, j’ai procédé à une
analyse transversale pour identifier les points saillants qui traversaient les
différents entretiens et positionner les argumentaires identifiés les uns par
rapport aux autres.


\chapter{Résultats et analyse}

\section{Typologie des propriétaires avant FOREDAVENIR}

Au cours des entretiens, plusieurs opérateurs ont décrit les propriétaires comme
étant peu actifs~; voir même, ils se distinguent par une~\textquote{absence de
culture forestière}. J'ai donc construit une typologie afin de confronter les
résultats de mes enquêtes avec la parole des opérateurs. Quatre catégories de
propriétaires se sont dessinées suite à ce travail de synthèse. Les 4 catégories
sont résumées dans le tableau~\ref{tab:typ}.

\begin{sidewaystable}%
  \centering%
  \caption{Classification des propriétaires}\label{tab:typ}%
  \begin{tabulary}{\columnwidth}{|p{28mm}|p{45mm}|p{45mm}|p{45mm}|p{45mm}|}%
    \hline Catégorie & Forestiers professionnels & Forestiers en devenir &%
    Agriculteurs & Propriétaires distants\\%
    \hline Discriminants&%
      \begin{compactitem}%
	\item très actifs%
	\item éléments moteurs de la filière locale%
      \end{compactitem}&%
      \begin{compactitem}%
	\item très actifs%
	\item sont en phase d'insertion dans la filière%
      \end{compactitem}&%
      \begin{compactitem}%
	\item actifs%
	\item ne sont pas inséré de manière construite dans la filière%
      \end{compactitem}&%
      \begin{compactitem}%
	\item inactifs%
	\item absence d'intégration dans la filière%
      \end{compactitem}\\%
    \hline Modèle de travaux&II;IE;EE&IE;IN&IN&NN\\%
    \hline Caractéristiques&%
      \begin{compactitem}%
	\item une grande propriété forestière >90ha%
	\item des blocs de parcelles cohérents%
	\item objectifs de production ou de défiscalisation%
      \end{compactitem}&%
      \begin{compactitem}%
	\item matériel propre pour le nettoyage et l'entretien%
	\item connaît le monde végétal%
	\item développe son patrimoine forestier selon une cohérence%
	\item considérations esthétiques ou biologiques%
      \end{compactitem}&%
      \begin{compactitem}%
	\item matériel propre pour le nettoyage et l'entretien%
	\item connaît le monde végétal%
	\item entretien du patrimoine forestier déjà existant
      \end{compactitem}&%
      \begin{compactitem}%
	\item ne possède pas de matériel%
	\item ne connaît pas ou mal ses parcelles%
	\item n'a que très peu été en contact avec le monde végétal%
      \end{compactitem}\\%
    \hline Bois-énergie & valorise au maximum les bois et ne produit le BE que %
      lorsque ce dernier n'est pas bon pour aller en BO ou en BI&%
      &coupe son propre bois bûche mais ne le commercialise pas%
      &fait couper ses bois pour avoir du bois bûche mais ne le commercialise pas\\%
    \hline
  \end{tabulary}
\end{sidewaystable}

\subsection{\textquote{Forestiers professionnels}}

Le groupe des \textquote{forestiers professionnels} se distingue par une forte implication
dans le domaine de la sylviculture et une intégration dans la filière. Ces
membres gèrent activement des propriétés souvent conséquentes, homogènes et
cohérentes. Leurs pratiques sylvicoles se caractérisent par un entretien des
parcelles et des récoltes de bois en adéquation avec leurs documents de gestion.
Ces propriétaires ont la capacité de définir leur mode de gestion, ce sur des
parcelles ou sur l'ensemble de la propriété. L'internalisation ou
l'externalisation des travaux d'entretien et des coupes découlent donc de leurs
choix de gestion. Certains de ses propriétaires utilisent ainsi
leurs connaissances et la surface de leurs forêts pour tester eux-mêmes des
essences et des itinéraires sylvicoles:

\blockquote[propriétaire 1]{C’est ma
station qui me conduit à mettre du pin maritime. Néanmoins, je m’amuse à
faire des essais. Donc c’est pour ça que j’ai mis un petit peu de
pin taeda, pas que du pin maritime. Donc j’ai déjà 2 parcelles en pin taeda, et
en cours de PSG j’en ferai une 3ème sur une autre zone. J’ai une parcelle de
chênes rouges d’Amérique, puisque le chêne rouge d’Amérique a à peu près la même
appétence de sol que le pin maritime sur les stations, et il se comporte bien.}

Tout en s’inscrivant parfois dans une logique expérimentale, les modes de
gestion adoptés par le groupe des \textquote{forestiers professionnels} ont toujours une
finalité économique. Les propriétaires cherchent ainsi à valoriser au maximum
leur propriété tant dans une optique de défiscalisation que de production de
bois. Cela signifie qu’ils sont intéressés par l’ensemble des produits bois
possibles, du bois-bûche en autoconsommation à la commercialisation du~\gls{bo},
du~\gls{bi} et~\gls{be}. Mettant en avant l’étendue et
la gestion active de leurs parcelles, certains affichent l’objectif de sortir
différentes qualités de bois à chaque récolte. Leurs capacités d'intervention ne
se bornent pas à la sylviculture. Les
\textquote{forestiers professionnels} utilisent leurs expériences professionnelles
et leurs connaissances sur la filière forêt-bois pour négocier au mieux la
commercialisation. Tel est le
cas de ce propriétaire qui fait jouer la concurrence afin d’obtenir les contrats
qui lui conviennent le mieux:

\blockquote[propriétaire 1]{Je fais une
consultation, je prends au mieux disant, pas forcément le
moins disant, je prends le mieux disant. Donc *** a travaillé sur mes
parcelles parce que ce sont eux qui avaient le marché pour le boisement
compensateur LGV. Là, actuellement je fais des boisements, je boise les
parcelles qui sont ici, donc là j’ai 6 ha en reboisement. Et donc là j’ai fait
l’exploitation du BE a été faite par ***, une entreprise du 79 ils m’ont
donné le meilleur prix à la tonne.}

\subsection{\textquote{Forestiers en devenir}}

Le groupe des forestiers en devenir se distingue par une forte implication dans
le domaine de la forêt et la volonté de s'intégrer dans une filière. Possédant
leur propre matériel, ces propriétaires s’investissent dans l’entretien et le
nettoyage de leur propriété. Souvent, la propriété est composée de parcelles formant des
blocs cohérents compris entre 3 et 20ha. Leur gestion sylvicole s’inscrit dans
une vision à long terme et intègre plus explicitement des considérations
esthétiques et environnementales sans pour autant négliger les fonctions
économiques de la forêt:

\blockquote[propriétaire 2]{Les bois, ils protègent les vignes. Alors des fois,
ils vous amènent certaines nuisances comme par exemple le manque d’ensoleillement,
des endroits un peu humides, mais lorsqu’il y a des périodes de gel, ça vous
arrête les courants d’air~[\ldots]~On sent bien que des fois
il y a 2 chênes qui se gênent, qu’il faudrait éclaircir les acacias
mais il faut quelqu’un qui puisse faire les coupes. Nous, économiquement, on serait bien contente de
faire une coupe de bois, de récupérer de l’argent avec les
acacias, faire peut être aussi une coupe de chênes parce que voilà ça les
entretiens. On a l’aspect environnemental mais on veut aussi le côté économique}


Ce témoignage illustre l’esprit des \textquote{forestiers en devenir}~: un intérêt
pour la dynamique des peuplements existants et un désir d’accroître leur
valorisation économique. Cet esprit se traduit ici par la recherche d’opérateurs
capables de faire les travaux de coupes. De manière plus générale, les
\textquote{forestiers en devenir} souhaitent développer leur patrimoine
forestier. Ils ont pu amorcer cette trajectoire avant FOREDAVENIR  puisque que
certains sont déjà engagés dans des démarches d’amélioration sylvicole quand
d’autres ont étendu leurs forêts par boisement ou rachat de parcelles adjacentes.
Construite sur le long terme, la stratégie de ces propriétaires est plus ou
moins avancée et peut prendre différentes formes, qu’il s’agisse de se doter
d’un~\gls{psg}, de se regrouper et d’adopter un~\gls{psgc}, de s’engager dans une
sylviculture type \indexp{Prosilva}  ou encore de transmettre la propriété forestière
avec une continuité de gestion.

\subsection{\textquote{Agriculteurs}}

Le groupe des~\textquote{agriculteurs} se distingue par une relative implication dans le
domaine de la forêt et par une absence d’intégration construite dans la filière.
Ces propriétaires sont agriculteurs ou ont connu le monde agricole de par leurs
parents. Ils connaissent donc le monde végétal et possèdent un petit parc
matériel qui leur permet d’entretenir les sous-bois. Leurs pratiques sylvicoles
se limitent le plus souvent à des activités de nettoyage comme l’explique un
propriétaire avant tout soucieux d’éviter~\textquote[propriétaire 3]{les broussailles} et d’avoir
\textquote[propriétaire 3]{des bois propres}. Mais cette logique d’action, qui se nourrit des héritages
agricoles du territoire, peut receler d’autres potentialités. Pour avoir une
propriété \textquote[propriétaire 3]{qui ressemble à quelque chose}, ce propriétaire aurait aussi aimé
regrouper ses parcelles dispersées sur plusieurs communes~: \blockquote[propriétaire 3]{mon seul regret
c’est ne pas avoir eu des parcelles un peu plus convenables}
Dans tous les cas, pour~\textquote{les agriculteurs}, la forêt apparaît comme un espace
complémentaire, un espace d’appoint. Ils récoltent régulièrement du bois bûche
pour leur consommation personnelle ou utilisent la forêt pour transformer
d’anciennes terres agricoles en un cadre de vie attractif.  S’ils peuvent aux
besoins couper quelques arbres, ces ventes très occasionnelles se font
généralement sans faire jouer la concurrence et sans contrat. C’est pour cela
que je dis qu’il y a une absence d’intégration construite dans la filière. 

\subsection{\textquote{Propriétaires distants}}

Le groupe des \textquote{propriétaires distants} se distingue par une absence d’implication
dans la gestion sylvicole et une absence d’intégration dans la filière. Ils
détiennent par héritage des propriétés généralement inférieures à 2ha et souvent
composées de parcelles éparpillées. Habitant parfois assez loin (sud Gironde,
Pays Basque\ldots{}), ils ne savent pas toujours où sont situées ces parcelles, n’ont
jamais vraiment eu de contact avec le monde forestier et considèrent
généralement que leur propriété n’est pas exploitable, ni même utilisable comme
espace d’appoint. Ce groupe se caractérise ainsi par une distance relationnelle
ou physique vis-à-vis de la forêt. Ce désintérêt peut aller jusqu’au rejet.
Certains de ces \textquote{propriétaires distants} perçoivent leur forêt comme
un fardeau~: \textquote[propriétaire 4]{en réalité, je voudrais m’en débarrasser} explique ainsi une
personne retraitée qui habite le sud du département de la Gironde.

On pourrait penser que ces personnes sont potentiellement intéressées par des
actions forestières ou foncières visant à restructurer leur propriété. Or, les
expériences vécues lors d’opérations de remembrement menées sur le territoire
dans les années 1980  semblent avoir plutôt laissé un goût amer. Tel est le cas
de cette propriétaire qui regrette \textit{a posteriori} d’avoir échangé deux petites
parcelles boisés en pin maritime contre une plus grande composée d’accrus~:
\blockquote[propriétaire 4]{rétrospectivement, j’aurais dû intervenir~; j’aurais dû refuser}
Si on peut s’interroger quant à l’influence de cette mauvaise expérience  sur
l’implication des personnes concernées dans un programme tel que FOREDAVENIR,
ce récit confirme surtout la difficulté de ces propriétaires à être acteur du
devenir de leur forêt.

La passivité des propriétaires n’est pas toujours aussi évidente et peut être
plus complexe. Elle peut être issu de la combinaison de choix personnels avec
des dynamiques forestières différenciées. Tel est le cas  d’un
propriétaire possédant 3ha en Est Gironde mais aussi plus de 100ha dans les
Landes. Tandis que celles-ci sont délégation de gestion totale, les parcelles
Girondines ne sont pas entretenues et ce propriétaire habite dans le pays Basque.
Même si la configuration est un peu particulière, il y a bien une forme de
distance physique et relationnelle avec la forêt.

\section{Appropriation de FOREDAVENIR par les propriétaires}

\subsection{Une appropriation à géométrie variable révélatrice de motivations hétérogènes}

Les propriétaires enquêtés n’ont pas tous la même perception du programme
FOREDAVENIR et n’expriment pas les mêmes attentes. Leurs motivations, les
leviers d’actions identifiés et les freins éventuels qui pourraient limiter leur
implication peuvent ainsi varier selon les propriétaires. Ce résultat peut
paraître logique au regard de leurs rapports différenciés à la forêt. Cependant,
les différentes formes d’appropriations identifiées ne recoupent pas
nécessairement les profils de propriétaires établis. Comprendre l’appropriation
à géométrie variable de FOREDAVENIR constitue donc une étape supplémentaire pour
appréhender les passerelles entre catégories et donc les dynamiques de
changement potentiellement impulsées par ce programme.\\

\begin{table}[ht!]
  \centering
  \caption{Matrice synthétique des interactions entre les actions offertes par
  FOREDAVENIR et les motivations des propriétaires}\label{tab:leitmotiv}
    \begin{tabulary}{\columnwidth}{|p{28mm}|p{28mm}|p{28mm}|p{28mm}|p{28mm}|}
    \hline \textbf{Catégories}&\textbf{Motivations / objectifs} & \textbf{Freins}
      & \textbf{Éléments attracteurs} & \textbf{Leviers pour mobiliser}\\
    \hline Forestiers professionnels&\multirow{2}{30mm}{Optimiser la production
      de biomasse}&Temporalité&Subventions&Amélioration\\
    \cline{1-1} Propriétaires distants&&Éligibilité&Faire exploiter par d'autres&Délégation\\
    \hline \multirow{2}{30mm}{Forestiers en devenir}&\multirow{3}{30mm}{Insertion filière}
      &Institutionnali\-sation&Obtenir des conseils&Regroupement\\
      &&Éligibilité&Changer d'opérateurs&Animation\\
    \cline{1-1}  Agriculteurs&&&Diagnostic&Amélioration\\
    \hline \multirow{2}{30mm}{Agriculteurs}&\multirow{3}{30mm}{Entretien du patrimoine}
      &Temporalité&Trouver des opérateurs&Animation\\
      &&Institutionnali\-sation&Diagnostic&Amélioration\\
    \cline{1-1} Propriétaires distants&&Éligibilité&Trouver un repreneur&Réorganisation foncière\\
    \hline
  \end{tabulary}
\end{table}

Le tableau~\ref{tab:leitmotiv} a permis de classer les motivations, les freins
perçus par les propriétaires, les éléments du programme qui les intéresse et les
leviers sur lesquels peuvent jouer les opérateurs pour inciter les propriétaires
à s'engager selon la typologie.

\subsubsection{Intensifier la production de biomasse}


Les \textquote{forestiers professionnels}, dont l’essentiel de la propriété est
généralement composée de pins maritimes, perçoivent dans FOREDAVENIR,
l’opportunité d’optimiser la conduite des peuplements
feuillus. Pour certains, il s’agit d’atteindre les objectifs de diversification fixés par
leur~\gls{psg}, pour d’autres il s’agit d’aligner la conduite des peuplements qu’ils
ont en Est Gironde sur la manière dont est géré le reste de leur propriété dans
le massif Landais. Cette dernière configuration concerne notamment une
\textquote{forestière professionnelle}. Elle souhaite installer des peupliers sur les
parcelles récemment achetées en Est Gironde, mais aussi un~\textquote{propriétaires distants}
qui espère faire exploiter et gérer sa propriété girondine (3ha environ) de
manière similaire à celle qu'il possède dans les Landes de Gascogne. Pour ces
propriétaires, la rentabilité économique des opérations constitue le
premier critère de décision~; il conditionne leur investissement dans
FOREDAVENIR Ils sont ainsi tout à fait disposés à exploiter
leurs taillis en~\gls{be} et ne voient aucune objection à utiliser les
futurs peuplements pour développer ce débouché, à condition qu’il soit
rémunérateur.\\ 

La plantation de feuillus est plus onéreuse que la plantation de résineux. La
rentabilité de peuplement feuillus est donc en partie dû à leur coût de
plantation. C'est pourquoi ces propriétaires montrent un intérêt particulier pour
les subventions à la conversion. Elles leurs permettront d'aligner la rentabilité
du feuillu sur celle du pin maritime. Parmi les obstacles rencontrés, on peut évoquer la difficulté
d’appréhender l’éligibilité des peuplements feuillus qu’ils possèdent
(voir \ref{sec:eli}). La
\textquote{lenteur} supposée du programme est autre frein mentionné. Conduisant leurs
peuplements selon un \indexp{PSG} très précis, certains \textquote{forestiers professionnels}
redoutent donc que le temps de montage, d’instruction des dossiers et que
les délais de versement des aides retardent les coupes inscrites dans leur
document de gestion.

\subsubsection{S'insérer dans une filière}

D’autres propriétaires, essentiellement des~\textquote{forestiers en devenir} et
quelques \textquote{agriculteurs}, attendent de FOREDAVENIR qu’il optimise leur
production de bois mais surtout qu’il favorise leur insertion dans une dynamique
de filière. Au-delà des travaux ponctuels qu’ils pourraient réaliser dans le
cadre du programme, ils cherchent ainsi à se constituer un réseau et à se
positionner dans une chaîne de valeur. Ce travail de prospection doit permettre
soit de trouver des opérateurs qui sachent les conseiller et faire les travaux
souhaités, soit de diversifier leurs interlocuteurs afin de sortir d’une
situation de dépendance vis-à-vis d’un opérateur donné. Dans cette optique,
ils peuvent également se montrer ouverts à des logiques de regroupement avec
d’autres propriétaires. Cet intérêt est d’autant plus marqué que ces
propriétaires, en plus d’être dans la même incertitude que la catégorie
précédente quant à l’éligibilité de leurs peuplements, peuvent être confrontés à
la difficulté d’atteindre la surface minimale requise : 4ha atteignable en lots
de 1ha dans un rayon de 5 km (voir~\ref{sec:eli}). 

Ces propriétaires sont donc
particulièrement sensibles à la dynamique d’animation développée dans
FOREDAVENIR tout en étant intéressés par les subventions. Cependant, ils apparaissent également vigilant quant à la nature et
aux modalités de cette dynamique collective. Attentifs aux arguments économiques
mais aussi environnementaux, ils ne souhaitent pas s’inscrire dans n’importe
quelle filière. N'étant pas issu du massif landais et possédant des propriétés
essentiellement composées d’essences feuillues, ils peuvent par exemple porter
un jugement critique sur la culture du pin maritime, en raison des risques
qu’ils lui associent (tempêtes, nématode, processionnaire, etc.). Ils ne sont donc
pas favorables à une extension de cette sylviculture et si une telle logique
institutionnelle devait émerger du programme, elle pourrait sans doute
constituer un frein à leur engagement. Défendant une conception plus
multifonctionnelle de leurs forêts, ces propriétaires veulent améliorer les
peuplements feuillus existants et développer une sylviculture permettant de
concilier production de bois, dynamiques écologiques et attachement à un
patrimoine. Leur rapport au BE s’inscrit dans cette perspective. Ils ne
s’opposent pas au débouché BE si celui-ci permet de valoriser les bois issus
des travaux de nettoyage ou des taillis mais ils ne
sont pas forcément convaincus par cet usage du bois en tant qu'objectif du
peuplement. Ils ne comptent donc pas adopter
une sylviculture permettant d’optimiser sa production.

\subsubsection{Évaluer et améliorer son patrimoine forestier ou foncier}

Certains propriétaires, généralement des \textquote{agriculteurs} et surtout des
\textquote{propriétaires distants}, ont une vision plus tâtonnante et restrictive de
FOREDAVENIR Ils connaissent peu de chose du programme, hésitent à y émarger et
dans tous les cas n’en font pas une priorité. L’intérêt éventuel qu’ils y voient
est qu’il peut leur permettre de faire évaluer leur patrimoine ou de réaliser
des travaux ponctuels à moindre coût. Ils sont donc principalement intéressés
par les subventions ou la réalisation d’un diagnostic sylvicole. Mais
l’implication de ces propriétaires reste bien hypothétique et, si elle était
effective, devrait être a minima.

La taille de la propriété est souvent le principal facteur de cette passivité.
Celle-ci constitue un double frein. D'une part, ils sont conscients que de petites
parcelles ne sont pas rentables, donc ils ne s'investissent pas dans la
sylviculture~; d'autre part, même si ils étaient motivés, ils ne pourraient pas
atteindre, seuls, les surfaces minimums requises.

Ce frein peut être d’autant plus rédhibitoire que ces propriétaires se montrent
généralement réticents à toute emprise institutionnelle : ils veulent rester
maîtres chez eux. Contrairement à la catégorie précédente, ils semblent donc
moins enclins à se regrouper. Considérant la forêt avant tout comme une
propriété privée qui fournit du bois-bûche et des champignons pour leur
autoconsommation, ces propriétaires craignent plus largement, en s’associant à
ce type de programme supervisé par des administrations de l’État, de se voir
imposer des travaux ou des documents de gestion qui remettraient en cause leur
autonomie. Cela ne signifie pas qu’ils n’ont pas de réseaux mais ceux-ci restent
informels (voisins, proches, collègues) et essentiellement tournés vers
l’échange d’informations.

Si d’autres propriétaires se montrent moins réticent quant à la dimension
institutionnelle de FOREDAVENIR, c’est parce qu’ils espèrent que ce programme
leur offrira l’occasion de se retirer de la forêt. Sans héritiers et considérant
que leur forêt est devenue une charge, certains des propriétaires enquêtés
envisagent en effet de vendre leur propriété et souhaitent utiliser FOREDAVENIR
pour améliorer sa valeur foncière et faciliter la recherche de repreneurs.

Enfin, la temporalité de décision de ces propriétaires n’est pas forcément en
adéquation avec la temporalité du programme. Alors que ce dernier nécessite de
déposer les dossiers dans un délai d’un an et demi après le début du projet,
les propriétaires consternés n’ont pas forcément conscience de ces échéances,
soit parce que la gestion de leur propriété est depuis longtemps « hors du temps »,
soit parce qu’ils ont d’autres priorités à ce moment de leur vie (maladie ou
maladie du conjoint notamment)~\citep{butler2017family_ref113}.  Même s’il n’est pas toujours
perçu ainsi, ce décalage temporel nous semble bien constituer un frein potentiel.

\begin{sidewaystable}%
  \centering%
  \caption{Répartition hypothétique des propriétaires de l'échantillon après FOREDAVENIR}\label{tab:typ2}%
  \begin{tabulary}{\columnwidth}{|p{28mm}|p{45mm}|p{45mm}|p{45mm}|p{45mm}|}%
    \hline Catégorie & Forestiers professionnels & Forestiers en devenir &%
    Agriculteurs & Propriétaires distants\\%
    \hline Nombre d'individus avant FOREDAVENIR &~3&~4&~6&~4\\
    \hline Nombre potentiel d'individus après FOREDAVENIR
      &\multicolumn{1}{p{45mm}@{\hspace*{\tabcolsep}\makebox[0pt]{$\longleftarrow$}}|}{~5}
      &\multicolumn{1}{p{45mm}@{\hspace*{\tabcolsep}\makebox[0pt]{$\longleftarrow$}}|}{~4}
      &\multicolumn{1}{p{45mm}@{\hspace*{\tabcolsep}\makebox[0pt]{$\longrightarrow$}}|}{~3}&~3\\
    \hline Discriminants&%
      \begin{compactitem}%
	\item très actifs%
	\item éléments moteurs de la filière locale%
      \end{compactitem}&%
      \begin{compactitem}%
	\item très actifs%
	\item sont en phase d'insertion dans la filière%
      \end{compactitem}&%
      \begin{compactitem}%
	\item actifs%
	\item ne sont pas inséré de manière construite dans la filière%
      \end{compactitem}&%
      \begin{compactitem}%
	\item inactifs%
	\item absence d'intégration dans la filière%
      \end{compactitem}\\%
    \hline Modèle de travaux&II;IE;EE&IE;IN&IE;IN&NN\\%
    \hline
  \end{tabulary}
\end{sidewaystable}

\subsection{Vers une recomposition de la typologie}

L’appropriation à géométrie variable de FOREDAVENIR et le fait que celle-ci ne
recoupe pas totalement la typologie établie permet de mettre en évidence la
manière dont les individus peuvent potentiellement se déplacer au sein de cette
typologie. C’est sur ces logiques de convergence que je vais maintenant
insister.\\

Le tableau~\ref{tab:leitmotiv} a permis de re-répartir les propriétaires au sein
de la typologie. On obtient donc une typologie identique à la première mais avec
une potentielle redistribution des individus dans les catégories. Le tableau
\ref{tab:typ2} rend compte d'une double dynamique au sein de l'échantillon.

D'une part des \textquote{forestiers en devenir} et des \textquote{agriculteurs}
auront accru leur taux d'activité tout en consolidant leur réseau; d'autre part,
quelques~\textquote{agriculteurs} et des \textquote{propriétaires distants}
auront distendu leurs relations à la forêt.\\

L'avènement officiel d'un regroupement sous forme de \gls{gieef}\index{GIEEF}
pour certains, modélisé par la
certification du \gls{psgc}\index{PSGc} par les services de l’État, les incitera à faire des
consultations pour obtenir les travaux qui valorisent au mieux leurs bois tout
en intégrant leurs considérations. A partir de ce moment, ils seront donc tenus
d'être actif, conformément à leur plan de gestion, et s'inséreront dans une
filière. Certes, l'officialisation du GIEEF sera probablement un élément
déclencheur, il ne faut cependant pas oublier les \textquote{forestiers en devenir}
qui ont une démarche moins collective. Ces derniers seront susceptibles de devenir
des \textquote{forestiers professionnels} mais sur un temps plus long et à
condition que leurs héritiers assurent une continuité de la démarche.

Les \textquote{forestiers en devenir} peuvent être suivis, dans cette démarche
d'accroissement du taux d'activité, par des \textquote{agriculteurs}. Étant
ouvert au monde de la forêt et soucieux d'exploiter leurs biens, forestiers ou
non, ils sont susceptibles de porter un intérêt croissant si on leur propose les
bons outils. Si le programme, de part de l'animation faite et les travaux
d'amélioration subventionnés, arrive à piquer la curiosité de ces
\textquote{agriculteurs}. Alors ils seront susceptibles de devenir des
\textquote{forestiers en devenir}. Cet \indexp{AMI} Dynamic bois semble donc
avoir l'effet escompté~: de part l'animation, il sensibilise
\textquote[\citealp{instructionTech_ref109}]{les propriétaires forestiers à
l'utilité de la gestion sylvicole} et met en
place de~\textquote[\citealp{instructionTech_ref109}]{nouveaux documents de gestion
durable dans le cadre exclusif de plans de gestions concertés}. Cependant, cette
tendance ne s'exprime pas chez tous les propriétaires.\\

Des propriétaires sans repreneurs ou consternés souhaitent profiter de FOREDAVENIR
pour augmenter la valeur du foncier. Certes, ils sont donc ouverts à la
réalisation de travaux mais pas dans l'objectif d'une mise en production à long
terme. L'objectif est de valoriser en vue d'une vente. Si ces propriétaires
venaient à vendre, ils auront d'une part, participé aux objectifs de mobilisation
des bois à destination des chaufferies~; d'autre part, ils auront utilisé l'AMI
pour, en quelque sorte, se désensibiliser à l'utilité de la gestion sylvicole.

\section{FOREDAVENIR, entre principes et ajustements}

La partie qui suit présente les déclinaisons opérationnelles de FOREDAVENIR afin
d’appréhender les dynamiques collectives et institutionnelles véhiculés par
le programme. Les outils et instruments mis en place constituent un prisme
privilégié pour étudier les effets d’une politique publique.
Ils influencent le comportement des individus et les interactions entre acteurs
d'une part, en orientant leur action et en formalisant une certaine représentation des
problèmes~; d'autre part, en créant des incertitudes~\citep{lascoumes2005_ref116}.
Si ces incertitudes peuvent être sources de confusions et de fragilisation, elles
témoignent également de la dimension exploratoire du programme et peuvent donner
lieu à des ajustements~; voir des innovations techniques ou organisationnelles. Les
résultats présentés ici n'ont pas vocation à être exhaustifs mais à illustrer cette
dynamique à partir de trois enjeux saillants.

\subsection{Tâtonnement autour des surfaces et des qualités des bois éligibles}\label{sec:eli}

\indexp{FOREDAVENIR} s'appuie sur 3 volets pour mobiliser rapidement du BE~:
le premier porte sur
l'investissement matériel dans la filière, le second porte sur l'investissement
pour l'amélioration des peuplements et le troisième sur l'animation. Les
propriétaires sont directement concernés par le deuxième volet. Comme dans les 2 autres
volets, pour être éligible aux aides, le demandeur d'aides doit respecter des
critères qui attribuent les financements selon les objectifs fixés. Les critères constituent
donc un des instruments permettant de traduire les principes du programme en
actions opérationnelles. La formulation des critères a en partie été délégué à
la~\gls{draaf}\index{DRAAF} afin que ces derniers s'adaptent au mieux au spécificités
régionales. Bien que cette délégation illustre le souci de prendre en compte les
réalités du terrain, elle ne résout pas pour autant tous les problèmes soulevés,
qu’ils s’agissent de l’opérationnalisation de notions parfois vagues
(ex~:~\textquote{peuplements dépérissant}) ou de la volonté de concilier de
multiples objectifs.\\

\begin{wraptable}{l}{0.5\textwidth}
  \centering
  \caption{Résumé des principaux critères d'éligibilité et de leurs ajustements}\label{tab:crit}
  \begin{tabulary}{\columnwidth}{|p{25mm}|c|c|}
    \hline Conditions d'éligibilité & Valeur de x & x ajusté\\
    \hline Dossier > x ha & 4&4\\
    \hline Dossier par blocs de x ha &1&1\\
    \hline Blocs dans un rayon de xkm &1&5\\
    \hline Après travaux, xt de bois doivent être du BE &x>0&0\\
    \hline 
  \end{tabulary}
\end{wraptable}

Le premier critère d'éligibilité est la surface. Initialement, le~\citet{instructionTech_ref109}
fixait le seuil minimum requis à la constitution du dossier à 4ha. Ces 4ha
pouvaient être répartis en îlots de 1ha sans distance maximale tant que ceux-ci
se situent sur la zone de FOREDAVENIR. Cette règle, qui reprend des critères
utilisés lors de l'attribution des aides aux parcelles sinistrées par Klaus,
montre la volonté de mobiliser la petite propriété tout en incitant au
regroupement. Cependant, à la différence de Klaus, pour inciter les petits
propriétaires à se regrouper, les~\glspl{ami} offre la possibilité aux
propriétaires de se regrouper manière informelle~:
\blockquote[administration publique 1]{L’objectif de regrouper les petites propriétés[…]
  là il est possible de regrouper les propriétaires sur une structure non
  formalisée. Donc franchement, ça c’est une première. Ça permet de sortir des
  dossiers avec des petites propriétés dans un rayon de 5km. Beaucoup plus de
  flexibilité, beaucoup plus}
Les propriétaires ne sont donc plus obligés de passer par des formes juridiques
complexes comme les ASLs (Associations Syndicales Libres) ou les ASAs (Association
Syndicales Autorisées). À la place, ces derniers doivent désigner un
mandataire ; il sera l'interlocuteur principal avec le service d'instruction et
sera chargé de redistribuer les aides qui incombent au dossier.

En résumé, tout en s'appuyant sur des règles existantes, ce programme
exploratoire cherche à tester de nouvelles solutions mais cette ambition
s’accompagne également d’hésitations et de tâtonnements. Peu après le lancement
du projet, les services d'instructions régionaux ont ajouté un critère
supplémentaire concernant la définition des surfaces éligibles~: les îlots
constitutifs du dossier devaient se trouver dans un rayon de 1km. Suite à des
discussions entre partenaires, cette distance a été augmenté pour atteindre 5km
(voir tableau~\ref{tab:crit}). Or, il est probable que des opérateurs avaient
déjà constitués des dossiers. Pour répondre aux nouvelles exigences, ces derniers
ont donc été obligés de reconstituer les dossiers afin de respecter les nouveaux
critères. Ces ajustements au début du programme peuvent donc ajouter aussi bien
une charge de travail supplémentaire pour les opérateurs qu'une certaine
incompréhension de la part des propriétaires qui ne comprennent pas pourquoi
leurs parcelles ne sont plus éligibles. Dans ces conditions, un opérateur
rencontré m'a confié que ces ajustements permanents pouvaient déstabiliser les
propriétaires au risque de discréditer l'opérateur auprès de ces derniers~: \blockquote[opérateur 1]
{Si c'est comme je pense, on va arrêter d'en faire. On passe pour des charlots!}.
Sur le fond, ces hésitations illustrent peut-être la difficulté de trouver un
juste équilibre entre 2 objectifs qui semblent s'opposer~: mobiliser des bois
chez les petits propriétaires tout en assurant aux opérateurs que leurs travaux
soient rentables. Un opérateur explique que même si le seuil minimum était fixé
à 4ha, son objectif est de~\textquote{graviter autour} des propriétés identifiées
pour essayer d'agglomérer un maximum de parcelles de manière à atteindre des
surfaces de travaux de 15ha.

Cette logique d’ajustement est également perceptible dans les débats autour du
tonnage en bois des parcelles. À un moment, il a ainsi été décidé que les
parcelles devaient contenir une quantité minimale et maximale de bois. Ce bornage
visait à favoriser la mobilisation des bois tout en évitant que les aides aillent
à des peuplements suffisamment riches pour s’auto-entretenir. Néanmoins suite à
de nouvelles discussions, la quantité minimale de bois ne fait plus partie des
règles d’éligibilités. Même si elle est en partie compensée par le durcissement
des règles sur les surfaces éligibles, cette suppression pourrait freiner
l’appétence des opérateurs. Mais on peut aussi penser que ce critère reste
officieusement en vigueur comme en témoigne ce propriétaire confronté à des
opérateurs qui refusaient un chantier par manque de bois~: \blockquote[propriétaire 5]
{On m'a dit~: \textquote{Ah mais il faut du volume, il faut au moins X bennes
pour amener l'outil, c'est très lourd, les charges}etc.}

Enfin, le dernier critère d’éligibilité concerne la nature des peuplements. Tout
en étant le c\oe{}ur du second volet, ce critère reste pour le moins ambigu.
Certes le~\citet{instructionTech_ref109}
défini comme peuplements éligibles les \textquote{taillis, de~\gls{tsf} ou d’accrus [\ldots],
avec un objectif d’amélioration en futaie} mais il reste flou sur l'état
sanitaire de ces peuplements. Il est écrit que ces
peuplements sont souvent vieillissants et dépérissant. Or, cette terminologie,
qui est très utilisée par les acteurs rencontrés et qu’on retrouve au cœur même
du dossier de présentation de FOREDAVENIR, apparaît plus délicate à définir et
à opérationnaliser. Les services de l’État sont d’ailleurs conscients que
\textquote[administration publique 1]{même
avec l’instruction technique, sur le terrain, c’est source d’appréciation~[\ldots]~Y
a sujet à discussion}. Cette logique d’appréciation
fragilise l’instruction des dossiers mais, d’une certaine manière, renforce aussi
la nécessité des échanges. Les agents des services de l’État se déplacent donc
sur les parcelles pour évaluer, avec les opérateurs et les propriétaires,
l’éligibilité des parcelles identifiées. Systématiques dans un premier temps,
ces visites pourront devenir plus ponctuelles dans un second temps lorsque les
différents cas de figures possibles auront été identifiés et permis une forme
d’apprentissage collectif.\\

\begin{figure}[t]
  \centering
  \includegraphics[width=\textwidth]{qAbattu.jpg}
  \caption{Photographie d'une parcelle inéligible}\label{fig:ineligible}
\end{figure}

Usantes et déstabilisantes, notamment pour les animateurs du projet, ces
différentes incertitudes peuvent témoigner d’un manque de cadrage initial. Mais
elles reflètent aussi la complexité du sujet et la dimension fondamentalement
exploratoire et expérimentale de FOREDAVENIR. Les ajustements constatés
peuvent ainsi être sources d’inventivité et révèlent l’importance des dynamiques
collectives au sein de chaque projet, comme le souligne un membre des services
de l’État rencontré. Mais on peut aussi relever que cette dynamique d’ajustement
accentue les écarts entre ceux qui sont investis pleinement dans le fonctionnement
projet et ceux qui en sont plus éloignés~; les propriétaires en sont un bon
exemple. Même pour
des \textquote{propriétaires professionnels}, l’appréhension  de certaines
subtilités est difficile. Tel est le cas de ce propriétaire qui avaient bien
consciences que seules les parcelles de feuillues étaient éligibles sans pour
autant savoir qu'ils devaient être sur pied(voir figure~\ref{fig:ineligible})~:

\blockquote[propriétaire 6]{Le bois énergie, on avait
  quand même un lot avec un ami, un lot assez important. Alors comme par hasard
  on avait coupé les bois mais ils étaient encore sur les lieux. Attention je
  vous dis bien les choses, le bois était coupé mais ils étaient encore sur les
  lieux, et ben il nous l'a refusé parce qu'ils étaient coupés! Voilà!}

\subsection{Enjeux de coordination, entre tensions et innovations}

\subsubsection{Redéfinition des rôles entre acteurs ou redéfinition des compétences?}

L’arrivée de nouveaux acteurs, de nouveaux référentiels et, plus largement, la
mise en place de chaînes d’approvisionnement propres au BE soulèvent des enjeux
de coordination entre les différents protagonistes de cette filière émergente
\citep{tabourdeau2014_ref117,BEbanosDehez_ref78}. La nécessité affichée de mobiliser
du \textquote{bois supplémentaire} pour accompagner le développement des chaufferies
tout en préservant la matière première du tissu industriel existant accentue les
défis posés. Cette demande croissante contraint en effet certains opérateurs
à faire plus de prospection et de
démarchage que ce que leur mission nécessite en théorie~: \blockquote[opérateur 2]
{Je ne vais pas dire que c’est un combat perpétuel mais c’est tout
le temps être à la recherche de nouveaux propriétaires}. Or, pour
cet opérateur, les difficultés rencontrées nécessitent d’améliorer les
complémentarités entre les différents maillons de la chaîne~: \blockquote{On pourrait
mobiliser plus de bois s’il y avait des échanges plus constructifs entre les
opérateurs, entre le public et le privé [\ldots] plus de coordination entre personnes
d’une même filière}. La question qu’on peut se poser est la
suivante : est-ce que le renforcement de la dynamique collective passe par la
mise en place de dispositifs facilitants les échanges ou est ce qu’il implique
une redéfinition de la mission des acteurs?\\

Ces enjeux de coordination sont au cœur même des \glspl{ami} puisque
\textquote[\citealp{instructionTech_ref109}]{l’objectif recherché par l’ADEME
est d’accompagner le financement d’actions
collaboratives entre les acteurs de la chaîne de mobilisation du bois
complémentaires aux dispositifs existants}. Dans cette logique d’innovation, le
développement du numérique et des nouvelles technologies de l’information offre
des opportunités que les porteurs de FOREDAVENIR ont saisi. Des
spécialistes du secteur de l’information géographique, comme l’IGN et le GIP
ATGeRI, ont donc été sollicités pour développer une plateforme informatique qui
facilite la circulation de l’information et les échanges entre partenaires du
projet. Cet outil ne vise pas seulement à compiler des données mais constitue aussi
une interface au c\oe{}ur de la dynamique collective. Elle accompagne les acteurs
tout au long de la démarche. Les diagnostics sylvicoles réalisés par le CRPF ou
le SIPHEM auprès des propriétaires sont donc déposés sur la plateforme. Selon
le choix fait par le propriétaire, ce diagnostic peut être ouvert à un ou
plusieurs des opérateurs partenaires, voir à l’ensemble d’entre eux. Autrement
dit, la plateforme peut aider les opérateurs privés à identifier plus facilement
\textquote{là, où il y a du bois exploitable}~; ce qui correspond à une des attentes des
acteurs rencontrés lors de l’enquête. On peut néanmoins noter que ce partage de
l’information, même s’il est contrôlé, peut aussi aviver des concurrences entre
prestataires et contribuer à redistribuer les cartes sur un territoire donné.
Également utilisé pour déposer et instruire les dossiers, la plateforme peut
aussi faciliter les échanges entre les opérateurs et les services de l’État.
Elle permet à ces derniers d’avoir connaissance quasiment en temps réel de
l’avancement des dossiers, d’accéder rapidement à un certain nombre
d’informations et donc, \textit{in fine}, d’assurer un meilleur suivi du projet. Si de
rares acteurs ont pu s’interroger sur les risques que faisaient peser un tel
contrôle pour la propriété privée, le fait est que les possibilités offertes par
la plateforme développées par le GIP ATGeRI ont conduit l’ADEME à préconiser son
utilisation au niveau national pour l’ensemble des projets \textquote{Dynamics Bois}.\\ 

Parallèlement, le bois-énergie et les différents dispositifs qui accompagnent
son développement contribuent à faire évoluer la compétence des opérateurs.
Dans le sillage de la tempête Martin (1999), certains opérateurs ont ainsi perçu
l’essor du bois-énergie comme une opportunité de se diversifier. Mais il a fallu
s’équiper et trouver de nouveaux process pour s’adapter aux exigences de ce
marché émergent~: 
\blockquote[opérateur 1]{Des essais on en a fait
  pendant 10 ans [\dots] On a senti un premier développement de chaudières. Là
  on a commencé à se dire qu’il fallait qu’on réagisse différemment, qu’on pense
  différemment le process et le bois qu’on va livrer. Pour la plupart, les
  petites chaudières, des collectivités qui avaient des besoin en qualité
  complètement différentes que ce qu’on faisait pour Facture. La qualité n’est
  pas la même, on a commencé à travaillé d’une manière un peu plus raisonnée
  pour adapter notre produit aux exigences des petites chaudières.}


Un programme tel que FOREDAVENIR contribue à cette logique d’équipement et de
monter en compétence. Pour pouvoir bénéficier d’aides à l’investissement matériels
ou être compétitifs par rapport aux autres prestataires, les opérateurs de
FOREDAVENIR doivent pouvoir démarcher les propriétaires, assurer le rôle de mandataire,
acquérir une certification ou encore diversifier leurs travaux. FOREDAVENIR
participe donc d’un processus de \textquote{professionnalisation}
avec un élargissement des compétences qui peut conduire à une certaine forme
d’autonomisation, c’est-à-dire à une volonté de conquérir des marchés de manière
autonome. Tel est le cas de cet opérateur, qui s’était plutôt spécialisé dans
les travaux de nettoyage après la tempête Martin mais qui souhaite désormais
s’investir aussi dans le reboisement pour pouvoir répondre aux nouvelles attentes
(FOREDAVENIR, projets de reboisements compensateurs….)~:
\blockquote{A l’époque, on travaillait avec **** pour faire des essais d’exploitation, de
rendement et ainsi de suite [\ldots] on fait du reboisement pour pouvoir répondre
aux AMIs [\ldots] on fait la préparation de sol. On travaille en partenariat avec
Planfor pour les plantations.}


Avec cet extrait, on voit que la redéfinition des compétences implique aussi
une redéfinition des relations entre acteurs de la gestion forestière. Cet
exemple illustre davantage
une logique de concurrence que de complémentarité. Il y a là une forme de
paradoxe~: un programme tel que FOREDAVENIR vise à améliorer les synergies
collectives au sein de la filière BE tout en favorisant d’une certaine manière
la mise en concurrences des opérateurs. \textit{A contrario}, le cadre réglementaire et
le respect des règles de concurrences peut aussi être un frein à l’évolution de
certaines compétences. Au cours de l’enquête, plusieurs opérateurs privés ont
conditionné l’accroissement de la mobilisation des bois à un démarchage plus
\textquote{pro-actif} des propriétaires. Or, par exemple, si les techniciens du CRPF
conseillent les propriétaires et peuvent leur donner une liste d’opérateurs,
ils nous ont aussi précisé qu’ils ne pouvaient pas orienter le choix de ce
dernier vers tel ou tel prestataire au risque de contrevenir à leur mission
d’ingénierie publique.

\subsubsection{L'importance des processus de capitalisation}

En devenant un~\textquote[administration publique 2]{instrument structurant des
projets~\textquote{Dynamic Bois}}, la plateforme développée par le GIP ATGeRI
démontre l'importance des processus de capitalisation. Au-delà de son rôle de
compilation de l'information, l'outil est en effet, lui-même, le fruit d'un
processus de d'expérimentation amorcé bien avant FOREDAVENIR~: \blockquote[administration
publique 1]{Le GIP ATGeRI, c’est une plateforme qui a été développé localement,
on l’avait pour le suivi de la reconstitution de Klaus.}

Tout en s'exprimant chez les services de l'État au travers de la plateforme, la
capitalisation s'exprime aussi chez le~\gls{siphem}\index{SIPHEM}. Dès 2004, le syndicat
entreprend une réflexion sur l'installation et l'approvisionnement des chaufferies du territoire~;
il y associe les propriétaires forestiers, les énergéticiens, les
collectivités et les ETF \citep{BEbanosDehez_ref78}. La continuité de cette réflexion a été assurée en
2011 par un \gls{pdm}\index{PDM} porté conjointement par le \indexp{CRPF}
et le SIPHEM. Cette ambition d'assurer une continuité
de l’information, de consolider les partenariats existants et de développer
un savoir-faire forestier s’incarne également dans la volonté de maintenir, grâce
aux financements de l’ADEME, un technicien forestier au sein du SIPHEM. Cette
dynamique se traduit par une meilleure connaissance du territoire et des
propriétaires, donc une meilleure relation entre porteurs de projets et
forestiers. Il en découle
et une meilleure réception de FOREDAVENIR par les propriétaires~:
\blockquote[propriétaire 2]{il connaît la propriété~[\ldots]~il arrive avec une
carte qui rassure~[\ldots]~je trouve que c'est bien, et puis on le connaît ici.
On a eu affaire a lui les uns et les autres, parce que si on a des travaux
forestiers chez soi, on s'est posé des questions, c'est un interlocuteur que
l'on connaît localement}


En résumé, FOREDAVENIR ne crée donc pas forcément de nouveaux outils mais
contribue à assembler et à étendre la portée des expériences déjà amorcées. Tout
en s'appuyant surde l'information capitalisée, FOREDAVENIR au processus de
capitalisation qui
pourra faciliter l'implémentation de futurs programmes en Gironde.

\subsection{Assurer la rentabilité des chaufferies, rémunérer le producteur~: un
équilibre économique difficile à trouver}

En contribuant à diversifier les prestataires sur le territoire et en facilitant
l'acquisition de compétences, FOREDAVENIR semble favoriser des logiques de
concurrence économique. Les partenaires du projet en sont
conscients. Un employé d'une collectivité à même rapporté au sujet d'un
opérateur en situation de monopole: de nouveaux opérateurs vont venir
\textquote[administration publique 3]{lui brouter la laine
sur le dos}. Un employé de l'opérateur en question affirme ne pas craindre cette
concurrence car d'une part, l'opérateur s'est équipé pour garder une longueur
d'avance sur de potentiels nouveaux concurrents~; d'autre part, l'interconnaissance
et sa connaissance du territoire le protégerait~: \blockquote[opérateur 3]{On
travaille tous ensemble, on se connaît tous, faut pas oublier.
Mais on ne se marche pas dessus. M.*** marque son territoire. Ce
n'est pas tout à fait ça mais}
A l'inverse, la venue d'une concurrence
économique est bien perçue par les~\textquote{forestiers en devenir}. Ils espèrent
qu'elle leur permettra de faire monter les prix du bois et de diversifier leurs
sources d'information.

\begin{figure}[h]
  \centering
  \includegraphics[width=\textwidth]{chaineDeValeur.pdf}
  \caption{Chaîne de valeur potentielle parmi les maillons de la filière bois-énergie}\label{fig:CDV}
  \floatfoot{Sources~: \citep{foretEntreprise233_ref107}}
\end{figure}

L'enjeu du prix d'achat des bois au propriétaire constitue une problématique à
laquelle le CRPF tente de répondre. Dans cette optique
le CRPF a produit un~\textquote{Mémento Aquitain du bois-énergie}
dans lequel il aborde la question de la maîtrise de la chaîne de valeur (voir
figure~\ref{fig:CDV}). Cependant, FOREDAVENIR n'impose pas de
prix chaudière. Le programme n'impose pas aux chaudières un prix minimum
susceptible de rémunérer l'ensemble des acteurs de la filière. L'\indexp{ADEME},
en tant que financeur d'un projet qui vise à sécuriser l'approvisionnement des
chaufferies, ne peut pas imposer un prix de la plaquette au risque de
déstabiliser l'équilibre économique des chaudières. Il en va de même pour le
\indexp{SIPHEM} si il veut remplir ces objectifs de \indexp{TEPOS}. La
transition énergétique ne peut s'effectuer que si le combustible bois reste
rentable face à ces concurrents, tel que le gaz ou l'électricité.
À ces contraintes du marché de l’énergie, les opérateurs forestiers rencontrés
ajoutent leurs propres charges (coûts de mobilisation, de transformation, de
transport du BE, etc.) pour expliquer leurs faibles marges de manœuvres vis-vis des
propriétaires~: \textquote[opérateur 2]{au prix de la chaudière. Et puis, faut que je me paye,
j'ai de comptes à rendre}. L’une des options pour réduire les coûts de production est de
privilégier le flux tendu. Il permet de limiter les ruptures de charge lié au
stockage et aux intermédiaires de production du BE. Le flux tendu est pertinent
pour approvisionner les grosses chaufferies industrielles qui consomment beaucoup
de BE et qui n’ont pas besoin d’un combustible de grande qualité. Les petites
chaufferies ont besoin d’un produit de meilleure qualité (taux d’humidité
moindre et granulométrie plus homogène). Elles justifient le recours à des
plateformes de stockage et des équipements de criblage. Elles permettent donc
aux opérateurs de créer de la valeur ajoutée. Elles permettraient
donc, in fine, de mieux valoriser le combustible bois. Cependant, elles
consomment de trop petites quantités pour créer un effet d’entraînement. Dans
ce contexte où le modèle économique du BE reste difficile à trouver, les
chaufferies de moyennes puissances (ex : les chaufferies collectives urbaines
type hôpital ou écoquartier) offrent donc ,en théorie, le rapport qualité /
quantité le plus intéressant pour les opérateurs. Même si ces chaudières
nécessitent \textit{a priori} une certaine qualité de combustible pour limiter les
risques de dysfonctionnement, leurs gestionnaires privilégient souvent des
combustibles à bas prix. Ce choix s’explique par des contraintes économiques
mais aussi par le constat d’un déficit d’homogénéité des approvisionnements
proposés \citep{BEbanosDehez_ref78}. Conscient de cette situation et de leurs
faibles marges de man\oe{}uvre pour intervenir dans une logique de marché, les
animateurs de FOREDAVENIR ont décidé de promouvoir le label \gls{cbq+}\index{CBQ+}. Créée en
Rhône-Alpes et portée par une association basée en Poitou-Charentes, cette
certification vise à garantir aux gestionnaires des chaufferies un combustible
calibré aux qualités adéquates et à pérenniser cet approvisionnement sur le long
terme. Pour le fournisseur, cette relation commerciale offre une visibilité et
l’opportunité d’une meilleure rémunération. Autrement dit, en cherchant à
développer ce label au sein de FOREDAVENIR, l’ADEME et les animateurs tentent de
reprendre en main l’enjeu de la contractualisation, d’influer sur les prix et,
\textit{in fine}, d’améliorer l’équilibre économique de la filière BE. Si
l’implémentation de cette certification et ses effets restent incertains,
l’expérimentation illustre une nouvelle fois l’esprit même des \glspl{ami},
entre principes et ajustements.

\chapter{Discussion et conclusion}

L'étude avait pour objectif de comprendre comment le programme \indexp{FOREDAVENIR}
pouvait \textquote{sensibiliser les propriétaires forestiers à l'utilité de la gestion
sylvicole}. Pour cela, nous avons observé le comportement des propriétaires à
une double échelle. D'une part, les propriétaires ont été analysés à l'échelle
micro~: individu par individu~; d'autre part, à l'échelle méso~: comment ces
derniers s'insèrent dans une filière composée d'acteurs et d'opérateurs. 
Les travaux décrits dans ce rapport ont répondu favorablement à l'hypothèse
qui avait été formulée sous forme de problématique~: Au-delà des objectifs
affichés, comment FOREDAVENIR va-t-il modifier les pratiques et les cadres de la
gestion forestière en Est-Gironde?

L'analyse micro a montré que le programme induit une double dynamique chez les
propriétaires. D'une part, des propriétaires se professionnalisent~; d'autre
part, des propriétaires cherchent à distendre leurs liens avec la forêt. L'analyse
méso, qui replace les propriétaires dans un contexte, a montré comment le
développement du BE et l'utilisation de nouveaux outils induisent une évolution
des relations entre acteurs et de leurs rôles. On assiste donc à l'émergence
d'un nouveau débouché sur lequel d'une part, les propriétaires tentent d'en
obtenir le meilleur prix~; d'autre part, les  opérateurs tentent de se
positionner.\\

L'article de \citet{brahic2017_ref114} conforte les résulats mis en avant par
l'enquête.
Selon eux, \textquote{les acteurs les plus à même de s'intégrer dans la filière
BE sont ceux qui disposent déjà d'un savoir-faire et d'une expérience en matière
de gestion forestière et de commercialisation des bois}. Cette affirmation étaye
les résultats observés au travers de la typologie. Dans les 2 cas, il existe un
fort contraste entre les \textquote{forestiers professionnels} et les
\textquote{propriétaires distants}. Vis à vis de l'article de \citet{brahic2017_ref114},
la typologie construite brosse le portrait de catégories intermédiaires. Le
travail présenté ici est donc complémentaire dans le sens où, tout en confirmant
les résultats de \citet{brahic2017_ref114}, il montre comment les individus
dans les catégories intermédiaires peuvent évoluer.
Cependant, ces évolutions sont à titre indicatif et n'ont pas valeur de prédictions.
Elles ne permettent pas de prédire
combien de propriétaires vont suivre ces dynamiques ni si leur évolution va
être rapide ou lente.
C'est pourquoi il sera intéressant de prolonger la démarche exploratoire par une
une enquête quantitative. L'enquête quantitative qui sera mise en place
en 2018 par le centre de recherche permettra de confirmer ou d'infirmer la
double dynamique observée. Elle permettra aussi de catégoriser les leviers et
les freins.
Tout en étant un élément qui entraîne des évolutions du comportement des propriétaires
et des opérateurs, FOREDAVENIR ne peut pas être considéré comme l'unique élément
déclencheur. Il est certes
un programme précurseur, mais n'est
toutefois que l'expression d'une évolution à l'échelle nationale des politiques
de récoltes de bois~\citep{sergent2014_ref110}. FOREDAVENIR ne fait donc
qu'accentuer les évolutions initiées par le développement du bois-énergie. Mon
stage a mis en évidence que la transition énergétique provoque une évolution des
relations entre acteurs et de leurs rôles sur 3 niveaux~: une évolution des
rôles et des missions des opérateurs de la filière, l'acquisition d'une capacité à peser
dans le domaine de la forêt chez des acteurs participant à la transition
énergétique et l'évolution du statut des propriétaires forestiers.\\

Le développement du BE et la transition énergétique constituent un nouveau marché.
En conséquence, les opérateurs investissent et s'équipent pour y conquérir des
parts. A ce titre, certains adhèrent à la certification CBQ+ pour se démarquer
de leurs concurrents alors que d'autres prennent part  à des études de faisabilité
sur des projets de construction de chaufferies.
La participation a des activités connexes aux métiers de la forêt impose aux
opérateurs un basculement de leur centre de gravité. Autrement dit, l'opérateur
ne se cantonne plus uniquement
au domaine de la forêt mais intègre des considérations énergétiques. En résumé,
ces 2 exemples montrent comment des exploitants forestiers acquièrent des
connaissances en énergétiques. L'étude laisse donc supposer que le développement
d'un nouveau marché, dans ce cas un nouveau débouché pour le bois, modifie en
profondeur le tissu économique ou industriel d'un territoire en redéfinissant le
rôle des opérateurs. L'impact du développement d'un nouveau marché n'est pas
spécifique au cas étudié~\citep{sergent2014_ref110} ni au secteur forêt-bois
\citep{castaneda2017_ref111}.

D'une part, comme il a été présenté précédemment, des opérateurs forestiers
prennent part à la transition énergétique; d'autre part, il est apparu, au cours
de l'enquête, que des acteurs de la transition énergétique capitalisaient de
l'information forestière. En capitalisant de l'information, grâce à la plateforme
ou grâce à une participation aux programmes successifs (\indexp{PDM} SIPHEM et
FOREDAVENIR), respectivement chez l'ADEME et chez le SIPHEM, octroie à ces acteurs
la capacité d'influencer l'action et les interactions entre opérateurs. D'une
manière plus générale, le gain d'une compétence forestière chez des acteurs de
la transition énergétique peut potentiellement faire évoluer le paradigme qui a
guidé l'action forestière jusqu'à aujourd'hui~\citep{sergent2014_ref110}. En
deçà de ces considérations, le SIPHEM s'est avéré être un excellent relais pour
implémenter FOREDAVENIR. Il semble que les propriétaires du SIPHEM aient été
plus réceptifs à l'esprit des AMIs tel que présenté dans~\citet{instructionTech_ref109}.
L'intérêt des collectivités dans les programmes d'incitation à la gestion forestière
a aussi été observé aux Etats-Unis~\citep{rouleau2016_ref100}.

En considérant les propriétaires comme le c\oe{}ur du programme, l'AMI pourrait
avoir comme objectif de donner une \textquote{culture forestière} aux propriétaires
en homogénéisant leurs pratiques sylvicoles. Cependant, \indexp{FOREDAVENIR}
n'utilise pas d'arsenal juridique pour~\textquote[\citealp{instructionTech_ref109}]
{sensibiliser les propriétaires forestiers à l'utilité de la gestion sylvicole}.
Autrement dit, il ne cherche pas à contraindre les propriétaires à appliquer un
seul modèle de gestion forestière. Le programme ne cherche donc pas à homogénéiser
les pratiques forestières. A la place, il propose aux propriétaires des éléments
(balivage, reboisement, régénération) qui permettront à chacun de participer à
l'objectif collectif du programme~; tout cela à la hauteur de leurs moyens. Sinon,
la 3$^{e}$ hypothèse est que le programme cherche à guider les propriétaires et
à les insérer dans une filière. Les propriétaires seraient encadrés par des outils
comme la plateforme. A l'extrême, un encadrement trop définis, grâce aux nouvelles
technologies, pourrait poser la question du respect de la propriété privée. Si
les objectifs de mobilisation ne pourraient pas être atteints par manque de
propriétaires mobilisés; l'IGN a proposé de repérer les peuplements privés pouvant
entrer dans le contexte de FOREDAVENIR. Avec cette information, le CRPF serait
en mesure de démarcher les propriétaires. Les propriétaires pourraient donc se
sentir dépossédés de leur droit de propriété privée puisque les services de
l’État connaîtraient \textit{a priori} les peuplements.\\

Bien que nous ayons réussi à appréhender le programme FOREDAVENIR dans son ensemble,
il n'a pas été possible de suivre l'intégralité des ramifications des volets
\textquote{investissement matériel et immatériel dans la filière} et animation
puisque nous nous sommes concentrés sur les propriétaires. Au travers du sujet
d'étude, nous avons donc été amené à nous concentrer sur le volet~\textquote{investissement
pour l'amélioration des peuplements forestiers}. Cependant, l'objet d'étude a
permis d'avoir un aperçu des enjeux de l'animation et de l'impact de la
mise en place d'un nouvel outil lié au volet~\textquote{investissement matériel
et immatériel}. Afin d'avoir une vision plus complète de FOREDAVENIR et des
dynamiques en cours, une ou des études pourraient analyser, en plus du volet
animation, la plateforme développée par le GIP ATGeRI. Cet outil, en tant que
n\oe{}ud d'échanges entre les acteurs, offrirait l'opportunité de mieux
comprendre les changements auxquels FOREDAVENIR contribue. L'analyse des
résultats et un agent des services de l'État suggèrent que les effets de FOREDAVENIR
vont au-delà des objectifs quantitatifs affichés. Ces travaux complémentaires
contribueraient donc à l'élaboration d'un dispositif d'évaluation.


\printbibliography[heading=bibliog]

\appendix

\chapter{FOREDAVENIR}\index{FOREDAVENIR}\label{App:appxA}
\pagenumbering{roman}
\newgeometry{top=-10mm,bottom=0mm,left=0mm,right=0mm}
\begin{figure}
  \centering
  \includegraphics[scale=1]{frda.pdf}
\end{figure}
\restoregeometry

%\chapter{Synthèse des enjeux et problématiques sur le territoire du Libournais}\index{Libournais}\label{App:appxB}
%\input{inputLibournais}

%\chapter{Synthèse des enjeux et problématiques sur le territoire du SIPHEM}\index{SIPHEM}\label{App:appxC}
%\input{inputSiphem}

\chapter{Grille d'entretiens des propriétaires}
\input{inputGrilleProprietaires}\label{App:appxB}

\chapter{Grille d'entretiens des acteurs économiques}
\input{inputGrilleActeursEco}\label{App:appxC}

\clearpage
\addcontentsline{toc}{chapter}{\protect\numberline{}Index}
\printindex

\newpage
\thispagestyle{empty}
~
\newpage

\newgeometry{top=0mm,bottom=0mm,left=-10mm,right=10mm}
\begin{figure}
  \centering
  \includegraphics[scale=1]{abstractIrstea.pdf}
\end{figure}
\restoregeometry

\end{document}
